\documentclass[10pt, draft, conference, letterpaper, twocolumn]{IEEEtran}

\title{EDP Symposium Brainstorming}
\author{
	\IEEEauthorblockN{
		Brent Nelson,
		Brad Riching, and
		Richard Black
	}
	\IEEEauthorblockA{
		Fulton College of Engineering and Technology \\
		Department of Electrical and Computer Engineering \\
		Brigham Young University, Provo, Utah 84602
	}
}

\begin{document}

\maketitle

\begin{abstract} 
	This paper philosophically explores the tradeoffs of using a text-based
	language versus a graphical tool to perform schematic capture of printed
	circuit board designs.  An enumeration of features that would be needed to make
	the text-based alternative favorable is given.
\end{abstract}

\begin{IEEEkeywords}
	PCB, HDL, schematic, design, EDA, CAD, layout, hierarchy
\end{IEEEkeywords}

\section{Introduction}
	\IEEEPARstart{T}{his} document serves as the breeding ground for and core
	dump of ideas regarding the philosphical implications of using a text-based HDL
	for circut board schematic capture as opposed to a graphical tool.

	The topics that will be explored are as follows:
	\begin{enumerate}[\IEEEsetlabelwidth{12}]
	  \item[1.] Historical Perspective
	  \item[2.] Asthetics Versus Content
	  \item[3.] Design Reviewability
	  \item[4.] Reusability
	  \item[5.] Collaboration
	  \item[6.] Development Speed
	  \item[7.] Features Needed for Normalcy
	  \item[8.] Features Needed for Superiority
	\end{enumerate}
	
\section{Historical Perspective}
	An exploration of the first logic-synthesis HDLs advent reveals several
	similarities to the introduction of a hypothetical circuit board language. 
	When ISP, KARL, and Verilog, et. al, were first invented, they were
	categorized as inefficient compared to and costlier than schematic
	capture.
	In modern industry, however, a digital engineer is considered gravely
	unqualified without proficiency in an HDL.
	
	This begs the following questions, (1) what caused HDLs to be unpopular at
	their inception, and (2) what caused HDLs to become streamlined among
	engineers.

	\ldots flesh out ideas here \ldots

	It follows, therefore, that the introduction of a circuit board HDL would
	provide similar benefits to those of logic-synthesis languages.

\section{Asthetics Versus Content}
	At the end of the day, the most important result of a design is its
	functionality.  In schematics, designs can become so complex that the layout of
	devices and wires becomes more important than the fundamental connectivity.
	This can shift the priority from functionality to asthetics. The vast majority
	of electrical engineers are hired for their inventive ideas and not their
	artistic ability.

	Through the use of a text-based schematic, the boundary between layout and
	design becomes well-defined; the primary goal becomes achieving the desired
	functionality. This will be demonstrated with a simple example: a ratsnest.
	
	\begin{figure}[h!]
		\centering
		\caption{Wire Ratsnests Schematic}
	\end{figure}
	In Figure 1 above, a collection of devices is interconnected in various ways.
	Assuming that the desired functionality has been achieved, the only problem
	with this schematic is the illegibility of its connectivity. If such a design
	is fully functional, there is little benefit in moving devices and modifying
	paths to make the drawing pleasing to the eye. Potentially, in the course of
	rearranging the various components, errors might inadvertently crop up due to
	the clean-up.
	
	Alternatively through an HDL, once proper connectivity is confirmed, there
	exists little benefit to perform revisions. The ratsnest is completely
	circumvented, and the overall design flow is expedited.
	
	All-in-all, a circuit board HDL eliminates the need to create a carefully
	organized drawing and instead focuses the engineer's attention on the
	connectivity and functionality of the design.  This does not limit those that
	prefer to see physical drawings of schematics as they are still free to pull
	out their pencils and sketch.

\section{Design Reviewability}
	On the major arguments in favor of graphical tools is the ability to visually
	inspect and review designs for proper functionality.  Design reviews are a
	necessary standard to prevent expensive mistakes by fabricating en masse a
	faulty circuit board. This is a potentially fatal flaw with an HDL approach to
	schematic capture.
	
	In order to rectify this flaw, it is wise to learn from the success of other
	languages, such as VHDL and Verilog. The question to be answered in examining
	these HDLs is how is a design review accomplished without visual schematics.
	
	One approach that VHDL and Verilog take to simplify design reviews is
	reductionism: the use of smaller modules that make up the greater. This allows
	a reviewer to verify proper behavior in manageable chunks. Such an approach
	would largely benefit a circuit board HDL.
	
	Another approach used by mainstream HDLs is the utilization of simulations.
	The reviewing typically consists of the plotting of several waveforms and
	checking for proper behavior at various points of interest.  In order to be
	maximally effective, a text-based schematic would need mixed-signal simulation
	capabilities.
	
	A unique and unconventional approach is to create and output rudimentary
	schematics to facilitate design reviews. These images need not be flawless;
	they must, however, convey enough information such that the verificiation is
	straightforward to somebody not familiar with the HDL.
	
	It is important to note that graphical tools are not exempt from scrutinization
	in this topic. One particular case where graphical tools make reviews difficult
	is the use of very high-pin-count devices.

	\begin{figure}[h!]
		\centering
		\caption{High-Pin-Count Device Schematic}
	\end{figure}
	In Figure 2 above, there is a single device in the center of the schematic that
	has so many pins that it fills the majority of the page.  As a result, the
	reviewer is required to turn several pages in order to perform verification. It
	is hard to debate that such a schematic is easier to review than a text-based
	one.
	
	In summary, the need for design reviews are a necessity and must be
	facilitated by the tools used to create the schematic.  If a circuit board HDL
	is to be successful, it must take from the features that make other languages
	easy to verify.  Such features include module compartmentalization and
	simulation.
	
\section{Reusability}
	A significant pitfall in the use of most graphical schematic capture tools is
	the inability to easily reuse working designs in others.  Often this process
	involves marqueeing the desired elements and copying them to the clipboard for
	use in another project.  Unfortunately the marquee tool does not exhibit
	the desired behavior; sometimes an entire device must be enclosed in order to
	be selected, other times only a portion must be in the selection area. 
	Furthermore, a straight copy and paste will also copy reference designators
	that might conflict with those in the newer design.  In either case, the
	engineer emerges the process with a splitting headache.
	
	An HDL would not have this problem since all attributes and connections can
	easily be grouped together and selected with ease. If the language included
	some method of autogenerating reference designators, the trial of modifying
	them upon design reuse becomes insignificant.
	
\section{Collaboration}

\section{Development Speed}

\section{Features Needed for Normalcy}

\section{Features Needed for Superiority}

\section{Conclusion}
	
\end{document}
